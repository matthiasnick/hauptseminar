\section{Outlook}

The results of microstucture modelling have shown that small strain rates result in large values for the dynamically recrystallized volume fraction. This effect is due to the limitations of using absolute thresholds for microstructural parameters. The models should be extended to give a more dynamic model which allows a bigger range of these paramenters.

Furthermore, a systematic approach to the comparison of various FEM packages with each other and with reality should involve, as a first step, the research of the capabilities of the involved packages and the development of an according model process. Also, an even more detailed process model should be set up, using the same time increments, mesh, boundary conditions etc.

Lastly, a validation of the results in a real process is necessary. This includes a validation of both the microstructural model with grain sizes measured during the process and the process parameters such as die forces and material temperature development.
