\section{Summary}

It has been shown that the microstructural modelling of multi-pass open-die forging processes is possible. However, problems arise at small strain rates due to high resulting values for the dynamically recrystallized volume fraction. These, however, can be show to stem from a numerical effect at very small strain rates. This leads to SRX kinetics during the second heat that seem unrealistic. This leads to grain sizes that have to be evaluated critically.

The validation of the results from DEFORM has proven to be difficult. Differences in the approach of the various packages towards kinematics definition result in problems in recreating the process. However, where comparisons can be made, the systems still yield different results. This is especially obvious in the thermal modelling of the heating process, which is independent from die kinematics but still shows different results. Moreover, the resulting lengthening during the forging process differs between the used FEM packages, even leading to different numbers of blows needed to reach forging of the entire work piece.

In comparison, modelling in DEFORM using the cogging tool is a very handy way to setup an open-die forging model. Modelling open-die forging processes using FORGE is generally easy to do as well. However, more individual modelling leads to a sharp increase in the required effort. PEP/LARSTRAN, on the other hand, needs the user to have a deeper understanding of FEM as a concept and requires far more work in defining die kinematics.
