\section{Einleitung}

In der Fertigung von Prototypen und Kleinserien ist die Anwendung traditioneller Blechumformverfahren wie des Tiefziehens oder des Streckziehens aufgrund der Notwendigkeit von hochfesten Werkzeugen und Maschinen, die hohe Kräfte aufbringen können, sowohl kosten- als auch zeitintensiv. Ein Prozess, der diese Beschränkungen umgeht, ist die inkrementelle Blechumformung (IBU), welche mit einem universellen Werkzeug, das Blech inkrementell umformt. Für die IBU ergeben sich jedoch einige Prozessgrenzen. Neben der langen Dauer des Prozesses und seiner schlechten Abbildbarkeit in Simulationen sind dies insbesondere die Neigung zu Ausdünnung, Einschnürung und schließlich Riss des Blechs bei hohen Wandwinkeln sowie die schlechte Geometriegenauigkeit. \cite{dissbambach,dissames}\par

Das Problem der Geometriegenauigkeit zeigt sich besonders bei schwerumformbaren Werkstoffen wie dem in der Luftfahrt oft verwendeten TiAl6V4. Dieses besitzt neben einem niedrigen Elastizitätsmodul auch eine hohe Fließgrenze, so dass ein großer elastischer Anteil an der Formänderung besteht. Dies wiederum führt zur Rückfederung des Werkstücks nach dem Umformen. Weiterhin ist das Umformvermögen von TiAl6V4 bei Raumtemperatur nur gering. Zur Überwindung dieser Prozessgrenzen wurden bereits Versuche unternommen, die Fließspannung des umzuformenden Blech durch Erwärmung zu senken. Neben der Erwärmung des gesamten Bleches während des Prozesses ist es möglich, den Werkstoff gezielt in der Umformzone zu erwärmen. Dies kann konduktiv oder mit Laserstrahlung erfolgen. \cite{hybridisf,diplbailly}\par

Zur laserunterstützten inkrementellen Blechumformung ist bereits eine Optik entwickelt worden, die es ermöglicht, den Laserbrennfleck um das Umformwerkzeug herum zu bewegen. Ebenfalls existiert bereits ein Programm, welches auf Basis eines bestehenden NC-Pfads zu einem IBU-Prozess die notwendigen Bewegungen des Laserbrennflecks bestimmt \cite{laseraisfti}. Es ist jedoch aus thermischen und geometrischen Gründen nicht sinnvoll, den Laser dauerhaft mit der gleichen Ausgangsleistung zu betreiben. Daher ist das Thema dieser Arbeit eine Vorausberechnung der benötigten Laserleistung unter Berücksichtigung der tatsächlichen Geschwindigkeit des Laserbrennfelcks und von Wärmeleitungs- und Wärmeverlustphänomenen sowie eine theoretische Validierung des dazu entwickelten Algorithmus und das Erstellen eines Konzepts für die Einbindung in das bestehende CAX-Umfeld.