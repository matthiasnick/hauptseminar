\section{Introduction}

Open-die forging is the oldest forging process and can be used to create a variety of final forms. It is an incremental, highly flexible metal forming process. The process typically involves two dies of simple geometry moving towards one another and thus forming the work piece. Open-die forging processes can be separated into four categories: upsetting, stretch forging, punching and hollow forging. This work, however, will focus on a stretch forging process.\cite{hbut}

Since the properties of the work piece depend strongly on the microstructure, methods to predict microstructural behaviour have been researched. The calculation of microstructure devolution based on the Finite Element modelling of the macroscopic mechanical process has proven to produce good results in this field. However, a systematic approach to test these results for open-die forging processes has yet to be undertaken. This work will focus on the microstructural modelling of a process that can easily be reproduced in reality.

Besides, the Finite Element results of the primarily used FEM package will be validated using other software. Software packages will be compared based on their usability and flexibility to describe open-die forging processes.

