\section{Foundations}

\subsection{Open-die forging}
Open-die forging is the oldest forging process and can be used to create a variety of final forms. It is an incremental, highly flexible metal forming process. The process typically involves two dies of simple geometry moving towards one another and thus forming the work piece. Open-die forging processes can be separated into four categories: upsetting, stretch forging, punching and hollow forging. This work, however, will focus on a stretch forging process.\cite{hbut}

The incremental and flexible nature of open-die forging makes it suitable primarily to the manufacturing of small lot sizes or for the forming of parts that cannot be produced by other processes due to power and force limitations of these processes. Its primary use is in the preparation of cast ingots for further machining. By open-die forging, cavities from the casting process can be rectified and the needed material properties can be reached.\cite{forgcomp}

\subsection{Finite element method}
The finite element method is a method to model, beside others, continuum mechanics of solid work pieces. The work piece is separated into discrete parts, called elements, which are themselves geometrically defined by nodes. While these nodes hold coordinates as information, the elements hold temperatures, stresses etc. Using material properties such as flow curves, friction, thermal conductivity and emissivity, the system's reaction to thermal and mechanical external loads can be calculated.

Due to the non-linear nature of the resulting equation system, only very simple models can be calculated analytically while most must be approximated numerically. Besides matters of usability, the numerical approach is the most important difference between available software packages.
